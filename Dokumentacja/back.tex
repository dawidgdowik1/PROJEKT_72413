
\chapter{Podsumowanie}

Celem zrealizowanego projektu było zaprojektowanie i implementacja systemu informatycznego wspomagającego kluczowe procesy w biurze podróży. Zakres zrealizowanych prac objął analizę wymagań funkcjonalnych, zaprojektowanie relacyjnej bazy danych oraz stworzenie aplikacji umożliwiającej zarządzanie ofertą turystyczną i bazą klientów.

Aplikacja została wykonana zgodnie z przyjętym harmonogramem. Główny nacisk położono na poprawność logiczną operacji biznesowych (np. rezerwacja wycieczki) oraz spójność danych. System pozwala na efektywną ewidencję klientów oraz obsługę procesu sprzedaży, co stanowi realizację głównego celu projektowego.

\section*{Planowane dalsze prace rozwojowe}
Projekt posiada architekturę otwartą na modyfikacje. W ramach dalszego rozwoju systemu zaplanowano następujące rozszerzenia:

\begin{itemize}
    \item \textbf{Moduł dostępu dla klienta (Web)} – stworzenie strony internetowej zintegrowanej z tą samą bazą danych, umożliwiającej klientom samodzielne przeglądanie ofert i dokonywanie wstępnych rezerwacji z domu.
    \item \textbf{Integracja z bramką płatności} – automatyzacja procesu księgowania wpłat poprzez podłączenie zewnętrznego dostawcy płatności (np. BLIK, PayU).
    \item \textbf{System powiadomień i marketingu} – automatyczne wysyłanie e-maili lub SMS-ów z przypomnieniem o terminie zapłaty lub informacją o nowych ofertach promocyjnych ("Last Minute") dopasowanych do preferencji klienta.
    \item \textbf{Rozbudowa modułu raportowania} – implementacja zaawansowanych wykresów statystycznych dla menedżera biura, obrazujących trendy sprzedaży w poszczególnych sezonach.
\end{itemize}

Stworzone oprogramowanie stanowi solidny fundament pod wyżej wymienione funkcjonalności.

\newpage
\addcontentsline{toc}{chapter}{Bibliografia}

\begin{thebibliography}{99}

\bibitem{Zeslawska}
E. Żesławska, \textit{Materiały do zajęć z przedmiotu Programowanie Obiektowe}, WSIiZ Rzeszów.

\bibitem{Fryc}
B. Fryc, \textit{Programowanie Obiektowe – materiały wykładowe}, WSIiZ Rzeszów.

\bibitem{Matulewski}
Matulewski J., \textit{C\#: lekcje programowania: praktyczna nauka programowania dla platform .NET i .NET Core}, Helion, Gliwice 2021 lub nowsze.

\bibitem{w3schools}
{W3Schools}, \textit{C\# Tutorial}, \url{https://www.w3schools.com/cs/} [dostęp: 04.02.2026].

\end{thebibliography} 


\newpage


\addcontentsline{toc}{chapter}{Spis rysunków}
\listoffigures
\newpage